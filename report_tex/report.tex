\documentclass{article}
\input{structure.tex} 
\usepackage{changepage}
\usepackage{soul}
\title{Weekly Report} % Title of the assignment

\author{Shanghui Huang} % Author name and email address

\date{\today} % University, school and/or department name(s) and a date
\begin{document}
\maketitle

\section*{Testing CXFEL Code with Multiple Dataset}
\subsection*{2BUK Imaging - Done}
Calculated squared distances between snapshots and projected onto diffusion map manifolds.

\subsection*{CroV Imaging - Done}
\noindent Applied mask and down sampled the raw data.

\noindent Calculated squared distances between snapshots and projected onto diffusion map manifolds.

\noindent Resorted along with the increasing order of first eigen values.

\noindent Using SnA method to calculate squared distance and dot product to get ATA.

\noindent Reconstructed two modes of topos and chronos.

\noindent Tested above process in both single node and MPI.

\noindent Finished a python master code, tested it with CroV data and successfully reproduced results of manifolds and chronos.

\subsection*{PYP Crystallography - Ongoing}
Using SnA method to caluclate squared distance

Read raw data into an NxD matrix - Done

Using SnA or SnA MPI function to write square or pipe block files - Done

\begin{adjustwidth}{\parindent}{}
Concatenate pipe files - Errors: Using numpy to initiate a 70Kx70K 2D array exceeds memory limits. Trying float16 datatype when Mortimer is back to normal.
\end{adjustwidth}

\st{Change data types can reduce size by $4$ at most, from double to half.}

\st{Cut whole sqDist distance before sparcify can work, tested with loop and CroV dataset.}

\st{Mortimer is back! But some environments has changed, still learning like how to activate conda, install VScode etc.}


\noindent Using SnA method to calculate dot product

\noindent Find ATA and extract chronos

\noindent With octave calling all python libraries, the code can run without bugs, but the result chrono seems off. Need to look into the transition between matlab and python code.

\noindent \textcolor{red}{Read PYP data and mask data in a correct format and test them with dSqM and dotM method in python master code}
\subsection*{GUI design - Ongoing}
A first version of quick and dirty GUI is done for diffusion map, details are going to be discussed.

\noindent {Added checkbox feature for \textit{h5} and \textit{transpose}, and pull down list feature for \textit{variable name}. Need to work on \textit{tab} function}

\noindent {Looking to add following new features to the GUI: Checkbox with true or false options (Done), tab close options (Done), displaying eigenvalues (Done), ssh connections (No need currently).}

\noindent {Added console window for the GUI, opened GUI on a remote server}

\noindent {Implemented the new python master code with NLSA function to GUI as a backstage code.}

\noindent \textcolor{red}{Modify GUIs with requests.}

\section*{Research Projects with Existing Code}
\subsection*{Data Incompleteness/Partiallity with NLSA Method}
Ongoing

\end{document}